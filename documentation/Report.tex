\documentclass[11pt, conference,letterpaper]{IEEEtran}
\IEEEoverridecommandlockouts

%
% ─── PREAMBLE ───────────────────────────────────────────────────────────────────
% \usepackage[a4paper, total={7in, 8in}]{geometry}
% \usepackage{fancyhdr}

% \fancypagestyle{plain}{%
%    \fancyhf{}
%    \fancyfoot[C]{\iffloatpage{}{\thepage}}
%    \renewcommand{\headrulewidth}{0pt}}
% \pagestyle{plain}

\usepackage[T1]{fontenc} % Use 8-bit encoding that has 256 glyphs
\usepackage{fourier} % Use the Adobe Utopia font for the document - comment this line to return to the LaTeX default
\usepackage[english]{babel} % English language/hyphenation
\usepackage{amsmath,amsfonts,amsthm} % Math packages
\usepackage{bm}
\usepackage{graphicx}
\usepackage{mathrsfs}
\usepackage{apacite}
\usepackage{float}
\usepackage{listings}
\usepackage{color}
\usepackage{enumitem}
\usepackage{tabularx}

\usepackage{mathtools}  
\mathtoolsset{showonlyrefs}  

\lstset{
    basicstyle=\footnotesize,
    columns=fullflexible,
    breaklines=true,
    tabsize=2,
    postbreak=\mbox{\textcolor{red}{$\hookrightarrow$}\space}
}

\graphicspath{{./images/}}

%
% ──────────────────────────────────────────────────────────────────── II ──────────
%   :::::: R E P O R T   O P E N I N G : :  :   :    :     :        :          :
% ──────────────────────────────────────────────────────────────────────────────
%
% \title{Advanced Algorithms Assignment 2}
% \author{Zaymon Foulds-Cook}

\begin{document}

\title{Documentation for 2811ICT Web Programming}
\author{\IEEEauthorblockN{Zaymon Foulds-Cook s5017391}
\IEEEauthorblockA{Griffith University School of Information Communication Technology}}

\maketitle


%
% ─── INTRODUCTION ───────────────────────────────────────────────────────────────
%   

\url{http://github.com/ZaymonFC/
WebProgramming-Assignment}

\section{Git Usage}
\subsection{Repository Organization}
\par The git structure of the git repository is simple, the top level directories of the angular project are at the top level of the repository along with the server directory and the documentation directory. The reduction in nesting resulting from not having the Angular app in it's own directory is ideal.
\subsection{Git Practices}
Throughout the development process git has been used heavily. The first section of the assignment consists of over 33 commits. A commit has been added to the repository either after each logical increment in development or when fixes are completed for bugs or unfinished features. Each commit has been used as a wrapper for a collection of related changes and the version history of the project reflects the progression and development of the system. Good descriptive commit messages have always been used in order to summarize the changes contained within. In regards to workflow a process did not have to be standardized due to the size of the developer team (1 member).

\section{Data Structures}
\par Various types have been used to represent different entities in the system. The types are:

\begin{itemize}
    \item \textbf{User} | Data type for a user entity.
        \begin{itemize}
            \item id: string
            \item username: string
            \item rank: string
            \item email: string
        \end{itemize}
    \item \textbf{Group} | Data type for a group entity. Used on the front-end for the group detail view. Contains nested collections of users and channel summaries.
        \begin{itemize}
            \item name: string
            \item id: string
            \item description: string
            \item users: User: List<User>
            \item channels: List<ChannelSummary>
        \end{itemize}
    \item \textbf{Group Summary} | Lighter weight type for group entity that does not require nesting of associated user objects.
        \begin{itemize}
            \item name: string
            \item id: string
            \item description: string
            \item users: list<string> (user ids)
            \item channels: list<string> (channel ids)
        \end{itemize}
    \item \textbf{Channel} | Type for the channel entity, used in the channel detail view. Contains nested users and a reference to the group it belongs to.
        \begin{itemize}
            \item id: string
            \item name: string
            \item users: list<User>
            \item groupId: string
        \end{itemize}
    \item \textbf{Channel Summary} | Lighter weight type for channel entity that does not require nesting of associated user objects.
        \begin{itemize}
            \item id: string
            \item groupId: string (id of group)
            \item users: list<string> (user ids)
        \end{itemize}
\end{itemize}

\section{Separation of Responsibilities - REST API and Client}
\subsection{API Responsibilities}
The responsibilities of the rest API are the concerns of data-access and persistence. When entities are created, modified or deleted on the client these changes must persist to be useful. For the client to be useful it must be able to present the user with data (hence the name presentation layer). To persist changes or fetch data the client must interact with the API. These interactions are managed by sending requests to controlled endpoints (routes) where the behavior of the server is unique to the action each route is responsible for. In this project the API exposes routes to fetch, update, create and delete all entities listed in section 2. The API makes use of a functional data layer to perform queries, search, select, delete and manage foreign keys on entities stored in separate JSON files. The functional data layer also manages keys and makes use of UUIDV4 for primary keys. The data layer can enforce unique fields.

\subsection{Client Responsibilities}
The responsibilities of the angular client are to present data to the user and to allow users to interact with the application in a real time and visual way. In this assignment the client is responsible for presenting the data as well as allowing the users to manage entities such as groups, channels and other users. The client in this assignment also has a granular permission system implemented as simple action guards to hide sets of functionality from users of different ranks.

\section{Routes}
\begin{itemize}
    \item \textbf{GET /login/:username} | A simple route which returns a user object if the user is found in the data otherwise returning null..
        \\ \textit{Parameters: }
        \begin{itemize}
            \item UserID: string
        \end{itemize}
        \textit{Outputs: }
        \begin{itemize}
            \item User Object
        \end{itemize}
    \item \textbf{GET /user} | Route to return the top 100 users (for user lists).
        \\ \textit{Parameters: }
        \begin{itemize}
            \item None
        \end{itemize}
        \textit{Outputs: }
        \begin{itemize}
            \item list<User>
        \end{itemize}
    \item \textbf{GET /user/:id} | Route to find and return a specific user.
        \\ \textit{Parameters: }
        \begin{itemize}
            \item UserID: string
        \end{itemize}
        \textit{Outputs: }
        \begin{itemize}
            \item User Object
        \end{itemize}
    \item \textbf{POST /user} | Route to create a new User entity.
        \\ \textit{Parameters: }
        \begin{itemize}
            \item username: string
            \item email: string
            \item rank: string
        \end{itemize}
        \textit{Outputs: }
        \begin{itemize}
            \item User Object
        \end{itemize}
    \item \textbf{PATCH /user/:id} | Route to update the contents of some user in the database.
        \\ \textit{Parameters: }
        \begin{itemize}
            \item UserId: string
            \item Object of changes expressed as key value pairs (key attribute name, value is the new value).
        \end{itemize}
        \textit{Outputs: }
        \begin{itemize}
            \item VOID (respond Status OK)
        \end{itemize}
    \item \textbf{DELETE /user/:id} | Route to delete a user and clear all reference to that user in the group and channel collections.
        \\ \textit{Parameters: }
        \begin{itemize}
            \item UserId: string
        \end{itemize}
        \textit{Outputs: }
        \begin{itemize}
            \item VOID (respond Status OK)
        \end{itemize}
    \item \textbf{GET /otherUsers/:id} | Route to fetch users that are not in a certain group (used for finding lists of users to add to groups).
        \\ \textit{Parameters: }
        \begin{itemize}
            \item GroupId: string
        \end{itemize}
        \textit{Outputs: }
        \begin{itemize}
            \item list<User>
        \end{itemize}
    \item \textbf{GET /group} | Route to get the top 100 groups (used for listing groups).
        \\ \textit{Parameters: }
        \begin{itemize}
            \item None
        \end{itemize}
        \textit{Outputs: }
        \begin{itemize}
            \item list<GroupSummary>
        \end{itemize}
    \item \textbf{GET /group/:id} | Route to find a specific Group, then queries the database and nests all User's associated with the group as nested objects. Also nests ChannelSummary's associated with the group.
        \\ \textit{Parameters: }
        \begin{itemize}
            \item GroupId: string
        \end{itemize}
        \textit{Outputs: }
        \begin{itemize}
            \item list<Group>
        \end{itemize}
    \item \textbf{POST /group} | Route to create a Group entity and store it in the database. Sanitizes input and creates a group with a scheme. Either returns 'non-unique' or the newly created Group object.
        \\ \textit{Parameters: }
        \begin{itemize}
            \item name: string
            \item description: string
        \end{itemize}
        \textit{Outputs: }
        \begin{itemize}
            \item Group: Group
        \end{itemize}
    \item \textbf{DELETE /group/:id} | Route to delete a group from the database. (Also deletes all associated channels).
        \\ \textit{Parameters: }
        \begin{itemize}
            \item groupId: string
        \end{itemize}
        \textit{Outputs: }
        \begin{itemize}
            \item Status OK
        \end{itemize}
    \item \textbf{POST /channel} | Route to create a new Channel. Sanitizes input information and then saves a new channel object to the database (includes reference to parent group). Updates group table to include a reference to the Channel. Returns newly created Channel.
        \\ \textit{Parameters: }
        \begin{itemize}
            \item name: string
            \item groupId: string
        \end{itemize}
        \textit{Outputs: }
        \begin{itemize}
            \item channel: Channel
        \end{itemize}
    \item \textbf{DELETE /channel/:id} | Route to  delete a channel and the parent groups reference to it.
        \\ \textit{Parameters: }
        \begin{itemize}
            \item channelId: string
        \end{itemize}
        \textit{Outputs: }
        \begin{itemize}
            \item Status OK
        \end{itemize}
    \item \textbf{PATCH /group/:id/user} | Route to add or remove users from group. 
        \\ \textit{Parameters: }
        \begin{itemize}
            \item methodName: string (add or remove user)
            \item groupId: string
            \item userId: string
        \end{itemize}
        \textit{Outputs: }
        \begin{itemize}
            \item Status OK
        \end{itemize}
    \item \textbf{GET /channel/:id} | Route to get a channel with nested users.
        \\ \textit{Parameters: }
        \begin{itemize}
            \item channelId: string
        \end{itemize}
        \textit{Outputs: }
        \begin{itemize}
            \item channel: Channel
        \end{itemize}
    \item \textbf{PATCH /channel} | Route to add and remove users from channels
        \\ \textit{Parameters: }
        \begin{itemize}
            \item method: string
            \item userId: string
            \item channelId: string
        \end{itemize}
        \textit{Outputs: }
        \begin{itemize}
            \item Status OK
        \end{itemize}
\end{itemize}

\section{Angular}
\subsection{Components}
\par Various components were created throughout development, some components form hierarchies of parent and child components. The app component is the top level component in the application. The app component in this application contains a simple template which include a router outlet and the menu bar. When the application navigates to other routes the router outlet will be populated with different components.

\begin{itemize}
    \item \textbf{Menu} | Menu component for logging out and navigation.
    \item \textbf{Home} | The home component contains the log in form for logging into the application.
    \item \textbf{Chat} | A simple component containing a basic sockets chat.
    \item \textbf{Dashboard} | Component for managing and viewing users and groups.
    \\ \textit{Nested Components}
    \begin{itemize}
        \item \textbf{dashboard-group} | Child list component which handles entering and deleting groups.
        \item \textbf{dashboard-user}  | Child list component which handles deleting, listing, promoting and demoting users.
    \end{itemize}
    \item \textbf{Group} | Detail view for a group that contains functions and functionality to create and remove channels, add and remove users from the group.
    \item \textbf{Channel} | Detail view for a channel that contains functions and functionality to add and remove users from the group.
    \item \textbf{NotFound} | Component to display if the route is not found.
    \item \textbf{Counter} | Simple component that displays a badge with the value of the number passed into the parameter.
\end{itemize}


\subsection{Services}
\begin{itemize}
    \item \textbf{User Service}
    \\\textit{Responsibilities}
    \begin{itemize}
        \item Manages user for current session
        \item Creates users
        \item Gets users
        \item Finds user
        \item Deletes users
        \item Promotes and demotes users to different ranks
    \end{itemize}
    \item \textbf{Channel Service}
    \\\textit{Responsibilities}
    \begin{itemize}
        \item Get channels
        \item Remove Channels
        \item Add user to channel
        \item Remove user from channel
    \end{itemize}
    \item \textbf{Group Service}
    \\\textit{Responsibilities}
    \begin{itemize}
        \item Get groups
        \item Delete Groups
        \item Find Groups
        \item Add user to group
        \item Remove user from group
    \end{itemize}
    \item \textbf{Permission Service}
    \\\textit{Responsibilities}
    \begin{itemize}
        \item Defines granular permissions
        \item Creates permission hierarchies based on role
        \item Can check whether the current user has a certain permission in their associated permission hierarchy
    \end{itemize}
\end{itemize}

\subsection{Modules}
\par The main module used during this assignment is the routing module, which defines the virtual routes for the application.

%
% ─── END ────────────────────────────────────────────────────────────────────────
%
\end{document}
